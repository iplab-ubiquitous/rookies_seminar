% Rookies-seminar レジュメ作成用TeXファイル
% 必要ファイル
% ・rookies.sty		(新人ゼミ専用スタイル)
% ・jumoline.sty		(改行許可する下線引きスタイル)
% ・multicolpar.sty	(途中段組変更スタイル)
%
%
% 2段組の使用法
% ----
% begin{multicolpar}{2}
%  This is a pen.(英語)
%
%  これはペンです.(日本語)
% end{multicolpar}
% ----
% 行空けで交互に出力できますが,
% 1パラグラフずつ,begin と endで囲んだほうが
% いいことがあります.
%
%
% アンダーラインの使用法
% ----
% \Underline{This is a pen.} \
% ----
% ・Uが大文字であること,
% ・尻尾のスペース記号"\ "(バックスラッシュスペース)
% に注意
%
%
% 図の挿入法
% \begin{figure}[ht]
%   \centering %中央配置
%   \includegraphics[width=0.3\columnwidth]{figure4.eps}
%   %\includegraphics[width=表示幅\columnwidth]{eps画像ファイル.eps} で読み込む
%   \begin{multicolpar}{2}
%     Hoge
%
%     ほげ
%   \end{multicolpar}
% \end{figure}
%
%
%

\documentclass[11pt,a4paper]{jarticle}
\usepackage{rookies} % rookies-seminar Style読み込み
\usepackage{float}


% --------------------1つ目のタイトル生成のための情報--------------------
%
\mytitle{title} % 発表論文タイトル
\mysection{1}{ABSTRACT} % {発表順}{発表セクション} 改行記号は”\\”で.
\mydate{YYYY}{M}{D}{水} % 発表年月日{西暦(4ケタ)}{月}{日}{曜日} 発表年月日
\myname{名前}{namae} % 氏名:{漢字}{ローマ字}
\mynumber{0} % 第○会新人ゼミ

% ----------------------------------------------------------------


\begin{document}
\mktitle % 自前のタイトル生成(MakeTitle)です.(1つ目のみ 20170426更新)

\begin{itemize}
    \item[出典:] 〜〜. 201x. 〜. IN {\it Proceedings 〜〜} (UIST '1x)
\end{itemize}

\hrulefill % 横線

\begin{multicolpar}{2}
\bf{ABSTRACT}

\bf{概要}

hogehogehogehogehogehoge

あああ

\end{multicolpar}

% --------------------2つ目以降のタイトル生成--------------------
% \newtitle{発表論文タイトル}{発表順}{発表セクション}
% 2タイトル目以降は手動で入力してくださいごめんなさい。自動で改ページされます。
\newtitle{title2}{2}{ABSTRACT} 

\begin{itemize}
\item[出典:] 
\end{itemize}

\hrulefill

\begin{multicolpar}{2}
\bf{ABSTRACT}

\bf{概要}

fugafugafugafugafugafuga

いいい

\end{multicolpar}


%3タイトル目
\newtitle{title3}{3}{ABSTRACT}

\begin{itemize}
\item[出典:] 
\end{itemize}

\hrulefill

\begin{multicolpar}{2}
\bf{ABSTRACT}

\bf{概要}

piyopiyopiyopiyopiyopiyo

ううう

\end{multicolpar}


\end{document}
